\documentclass{report}

\usepackage[T1]{fontenc}
\usepackage[french]{babel}
\usepackage[utf8]{inputenc}




\begin{document}
\title {FSAB 1302 - Chimie et Chimie Physique 2 - CM2}
\date{22/09/2014}
\author{Damien Deprez}

\maketitle
\tableofcontents

\chapter{Les gaz parfait}
\section{Théorie de Bernoulli}

Sur base d'une intuition, on peu exprimer la pression $p=\frac{1}{3}\rho \bar{c^2}$
$p=\frac{1}{3} \sum{}{}{n_i*m_i*\bar{c^2}}=\frac{1}{3}(\rho_1\bar{c_1^2}+...+\rho_i\bar{c_i^2})$

\section{Loi de Boyle}
$pV=cst=f_1(m,T) => V=\frac{f_1(m,T)}{p}$
\section{Loi de Charles}
$\frac{V}{T}=cst=f_2(m,p) =>  V=Tf_2(m,p)$\\
$\frac{pV}{T}=f_3(m)=\frac{f_1(m,T)}{T}=pf_2(m,p)$
$$pV=nRT$$
\section{Loi de Boyle et Charles et Hypothèse d'Avogadro}
$pV=nRT=nN_Ak_BT=Nk_BT$
$p=\frac{1}{3}\frac{mN}{V}\bar{c^2}=>pV=\frac{1}{3}mN\bar{c^2}$
$$\frac{1}{2}m\bar{c^2}=\frac{3}{2}k_BT$$
De cette relation, on remarque que le Kelvin ne peut pas être négatif. Il est donc borné (car m>0). Si l'énergie cinétique est nulle, la température est nulle $=>$ Définition du Kelvin\\
!!! La théorie des gaz parfaits n'est valable uniquement à faible pression !!!

\chapter{Distribution de Maxwell}
\section{Hypothèse}
\begin{itemize}
\item le milieux gazeux est homogène(pas dépendance à la position)
\item le milieu est isotrope (pas dépendance à la direction)
\item il existe une distribution de la vitesse (au sens statistique)
\item les propriétés macroscopique reflètent 
\end{itemize}
\section{Distribution de la vitesse en 1D et en 3D}
%insérer graphe des notes
Si on dérive la fonction mathématique de la distribution de vitesse, on obtient une courbe de Gauss.
\subsection{Notation}
$\phi$=forme mathématique de la distribution de vitesse\\
$f=\frac{d\phi}{dc}$=dérivée de la forme math de la distribution de vitesse par rapport à la vitesse
\subsection{espace de Vitesse}
$$\int_{-\infty}^{\infty}{\int_{-\infty}^{\infty}{\int_{-\infty}^{\infty}{f_1(c_1)f_2(c_2)f_3(c_3)dc_1dc_2dc_3}}}=1$$
Si le milieu est isotrope
$$F(c)=f_1(c_1)f_2(c_2)f_3(c_3)$$
$$\ln{F(c)}=\ln{f_1(c_1)}+\ln{(f_2(c_2)+f_3(c_3))}$$
$$\frac{\delta\ln{F(c)}}{\delta c_1}_{c_2,c_3}=\frac{d\ln{f_1(c_1)}}{dc_1}$$
$$\frac{d \ln{F(c)}}{dc}\frac{\delta c}{\delta c_1}_{c_2,c_3}=\frac{d\ln{f_1(c_1)}}{dc_1}$$
$$\frac{1}{c} \frac{d\ln{F(c)}}{dc}=\frac{1}{c_1}\frac{d\ln{f_1(c_1)}}{dc_1}=\frac{1}{c_2}\frac{d\ln{f_2(c_2)}}{dc_2}=\frac{1}{c_3}\frac{d\ln{f_3(c_3)}}{dc_3}=-\gamma$$
$$f_1(c_1)=Ae^{\frac{-\gamma}{e}c_1^2}$$
Comme c'est valable pour $c_1$, $c_2$, $c_3$, les trois variables sont indépendantes. Dès lors, cela doit être égal à une constante (négative ou positive ?). Physiquement, on ne sait pas prouver que cela vaut $-\gamma$ avec $\gamma \in R^{e^{+}}$.

Il nous reste à trouver $A$ et $\gamma$
Propriété:
\begin{itemize}
\item $\int_{-\infty}^{\infty}{f_1(c_1) dc_1}=1$
\item $\frac{1}{2} m\bar{c^2}=\frac{3}{2}k_BT$
\end{itemize}
Or $\int_{-\infty}^{\infty}{e^{-ax^2}dx}=\sqrt{\frac{\Pi}{a}}$\\
$\int_{-\infty}^{\infty}{x^2e^{-ax^2}dx}=\frac{1}{2a}\sqrt{\frac{\Pi}{a}}$
$$\bar{c_1^2}=\int_{-\infty}^{\infty}{c_1^2f_1(c_1)dc_1)}$$
$$=>f(c_1)=\sqrt{\frac{m}{2\Pi k_B T}}e^{-\frac{mc_1^2}{2k_BT}}$$
$$F(c)dc=f_1(c_1)f_2(c_2)f_3(c_3)dc_1dc_2dc_3 = 4\Pi c^2\frac{Mm}{2\Pi}^{\frac{3}{2}}e^{-\frac{Mm}{2RT}c^2}$$
avec $dc_1dc_2dc_3=4\Pi c^2 dc$
Valeur intéressante :
\begin{itemize}
\item vitesse la plus probable ($c_p=\sqrt{\frac{2RT}{Mm}}$)
\item vitesse moyenne ($\bar{c}=\sqrt{\frac{4}{\Pi}}c_p$)
\item vitesse moyenne quadratique ($\sqrt{\bar{c^2}}=\sqrt{\frac{3}{2}}c_p$)
\end{itemize}
On peu mesurer la vitesse des molécules et cela a permis de vérifier la distribution de vitesse
\chapter{L'effusion d'un gaz}
\section{Définition}
c'est le passage des molécules au travers d'une section de petite taille.
Cela se fait sans choc entre les molécules près de la paroi et pas de choc avec la paroi. => effusion est à comparer à la diffusion qui implique des chocs. 
Le taux de collision avec la surface de passage donne le débit de molécules
$$\frac{dN_e}{dt}=\frac{N}{V}A\int_{0}^{\infty}{c_xf(c_x)dc_x}=\frac{N}{V}A \frac{k_BT}{2\Pi m}^{0,5}=A\frac{p}{(2\Pi k_B Tm)^{0,5}}$$
A température fixée, les molécules les plus légères effusent plus rapidement, elles passent plus facilement. C'est donc un moyen de filtrer mais imparfait.
$$\frac{dN_1/dt}{dN_2/dt}=\sqrt{\frac{Mm_2}{Mm_1}}$$
C'est basé sur la distribution de vitesse d'un gaz
 
\chapter{Limite de la Théorie de Gaz parfait}
\section{Énergie d'un gaz monoatomique}
Elle est prédite correctement. Lorsque l'on calcule la capacité calorifique pour des gaz polyatomique, ce n'est plus le cas. La capacité calorifique varie avec la température.
On va devoir rajouter des possibilités de stocker l'énergie dans des degrés de liberté (translation(3) + rotation(3) + vibration).
Capacité calorifique = $\sum$ddl/2.
Molécule diatomique à température ambiante : $C_{V,m}=\frac{5}{2}R [J/mole*K]$
Molécule monoatomique à température ambiante : $C_{V,m}=\frac{3}{2}R$
\end{document}